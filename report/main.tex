\documentclass[a4paper, 12pt]{extarticle}
\usepackage[russian]{babel}
\usepackage[T1]{fontenc}
\usepackage{fontspec}
\usepackage{indentfirst}
\usepackage{enumitem}
\usepackage[
    left=20mm,
    right=10mm,
    top=20mm,
    bottom=20mm
]{geometry}
\usepackage{parskip}
\usepackage{titlesec}
\usepackage{xurl}
\usepackage{hyperref}

\hypersetup{
    colorlinks=true,
    linkcolor=black,
    filecolor=blue,
    urlcolor=blue,
}

\renewcommand*{\labelitemi}{---}
\linespread{1.5}
\setmainfont{Times New Roman}

\renewcommand{\baselinestretch}{1.5}
\setlength{\parindent}{1.25cm}
\setlength{\parskip}{6pt}

\setlength{\parindent}{1.25cm}
\setlist[itemize]{itemsep=0em,topsep=0em,parsep=0em,partopsep=0em,leftmargin=1.55\parindent}
\setlist[enumerate]{itemsep=0em,topsep=0em,parsep=0em,partopsep=0em,leftmargin=1.55\parindent}

\renewcommand{\thesection}{\arabic{section}.}
\renewcommand{\thesubsection}{\thesection\arabic{subsection}.}
\renewcommand{\thesubsubsection}{\thesubsection\arabic{subsubsection}.}

\titleformat{\section}
{\normalfont\bfseries}{\thesection}{0.5em}{}

\makeatletter
\renewcommand*\l@section{\@dottedtocline{1}{1.5em}{2.3em}}
\makeatother

\begin{document}

\begin{titlepage}
    \vspace{0pt plus2fill}
    \noindent

    \vspace{0pt plus6fill}
    \begin{center}
        \textbf{\large{Санкт-Петербургский национальный исследовательский университет
                информационных технологий, механики и оптики}}

        \vspace{0pt plus3fill}

        \textbf{\LARGE{РЕФЕРАТ}}

        \vspace{0pt plus1fill}

        \textbf{\Large{Биологическое загрязнение, его \\ последствия и способы предотвращения}}

    \end{center}

    \vspace{0pt plus8fill}
    \begin{flushright}
        Студент: \\
        \textit{Швалов Даниил Андреевич}

        \textit{Факультет ИКТ}

        Группа: \textit{К32211}

        Преподаватель: \\
        \textit{Ляшенко Оксана Александровна}
    \end{flushright}

    \vspace{0pt plus4fill}
    \begin{center}
        {Санкт-Петербург ~--- 2023}
    \end{center}
\end{titlepage}

\tableofcontents
\newpage

\section{Введение}

Человек всегда использовал окружающую среду в основном как источник ресурсов, однако в течение очень длительного времени его деятельность не оказывала заметного влияния на биосферу. Лишь в конце прошлого столетия изменения биосферы под влиянием хозяйственной деятельности обратили на себя внимание ученых. В первой половине нынешнего века эти изменения нарастали и в настоящее время лавиной обрушились на человеческую цивилизацию. Стремясь к улучшению условий своей жизни, человек постоянно наращивает темпы материального производства, не задумываясь о последствиях. При таком подходе большая часть взятых от природы ресурсов возвращается ей в виде отходов, часто ядовитых или непригодных для утилизации. Это создает угрозу и существованию биосферы, и самого человека. И биологическое загрязнение --- одна из таких угроз.

Цель работы --- определить, что такое биологическое загрязнение, чем опасно и как можно его предотвратить.

Задачи:
\begin{itemize}
    \item дать определение понятию биологического загрязнения и загрязнения в целом;
    \item показать актуальность проблемы биологического загрязнения;
    \item показать последствия биологического загрязнения;
    \item предложить способы предотвращения и устранения последствий биологического загрязнения;
    \item привести реальные примеры биологического загрязнения.
\end{itemize}

\section{Загрязнения}

\textbf{Загрязнение} --- это привнесение в среду или возникновение в ней новых, обычно не характерных для нее физических, химических, информационных или биологических агентов, или превышение в рассматриваемое время естественного среднемноголетнего уровня (в пределах крайних колебаний) концентрации перечисленных агентов в среде, нередко приводящее к негативным последствиям.

Загрязнению подвергаются различные сферы Земли: атмосфера, гидросфера и мировой океан, литосфера, педосфера, ближний космос, а также непосредственно среда жизни человека --- города и другие населенные места.

Любые вещества, химический, биологический вид (преимущественно микроорганизмы), физический или информационный агент, попадающие в окружающую среду или возникающие в ней в количествах, выходящих за рамки обычного содержания предельных естественных колебаний или среднего природного фона в рассматриваемое время, называются \textbf{загрязнителями}. Они делятся на:
\begin{itemize}
    \item химически стойкие (неразлагающиеся), не входящие в состав естественного круговорота веществ, а потом очень медленно разрушаются в окружающей среде, нередко аккумулирующиеся организмами в пищевых цепях;
    \item загрязнители, разрушаемые биологическими процессами, входящие в естественные круговороты веществ, и потому быстро исчезающие или подвергающиеся разрушению биологическими агентами в искусственных системах очистки (например, на станциях очистки сточных вод).
\end{itemize}

Непосредственными объектами загрязнения (акцепторами загрязняющих веществ) служат основные компоненты экотопа (местообитание биотического сообщества): атмосфера, вода, почва. Косвенными объектами загрязнения (жертвы загрязнения) являются составляющие биоценоза – растения, животные, микроорганизмы.

Загрязнение можно разделить по видам загрязнителей на несколько групп:
\begin{itemize}
    \item аэрозольные;
    \item биологические;
    \item визуальные;
    \item космический мусор;
    \item механические;
    \item микробиологические;
    \item радиоактивные;
    \item световые;
    \item тепловые;
    \item физические;
    \item химические;
    \item шумовые;
    \item электромагнитные.
\end{itemize}

\section{Биологическое загрязнение}

\textbf{Биологическое загрязнение} --- это привнесение в экосистемы в результате антропогенного воздействия нехарактерных для них видов живых организмов (бактерий, вирусов и др.), ухудшающих условия существования естественных биотических сообществ или негативно влияющих на здоровье человека.

Основными источниками биологического воздействия являются:
\begin{itemize}
    \item бытовые стоки, канализации предприятий пищевой промышленности (сахарные, сыроварные, молочные заводы, мясокомбинаты и т. п.);
    \item бытовые и промышленные свалки;
    \item кладбища;
    \item канализационная сеть;
    \item поля орошения;
    \item распространения инфекций по причине биокатастроф, возникших естественным путем (например, пандемии чумы, эпидемии, холеры, натуральной оспы, сыпного тифа);
    \item аварии на биологически опасных объектах (биозаводы, НИИ и др.);
    \item экологически опасная техногенная деятельность (выемка грунта, добыча полезных ископаемых, исследование Крайнего Севера, сопряженное с извлечением из недр Земли древних бактерий и других организмов);
    \item неконтролируемая техногенная деятельность (селекция и отбор антибиотикоустойчивых патогенных штаммов микроорганизмов);
    \item природные катастрофы (сели, наводнения, цунами, приводящие к вспышкам природной заболеваемости);
    \item выбросы предприятий биологического (микробиологического) синтеза;
    \item бактериологическое оружие;
    \item продукты разложения тех отходов, которые отравляют окружающую среду и заражают её;
    \item инфицированные животные.
\end{itemize}

Биологическое загрязнение может быть случайным или связанным с деятельностью человека. Оно проявляется через проникновение в эксплуатируемые экосистемы и технологические устройства чуждых им растений, животных и микроорганизмов. Наиболее подверженные биологическому загрязнению части экосистемы: вода и почва. При отсутствии прямого влияния на живые организмы заражение происходит через пищу или контакт со средой обитания.

Разновидностью биологического загрязнения является микробиологическое загрязнение. Особенно загрязняют среду предприятия, производящие антибиотики, ферменты, вакцины, сыворотки, кормовой белок, биоконцентраты. Другими словами, это предприятия промышленного биосинтеза, в выбросах которых присутствуют живые клетки микроорганизмов. К биологическому загрязнению можно также отнести случайную и преднамеренную  интродукцию видов, чрезмерную экспансию живых организмов, то есть введение в культуру дикорастущих растений, распространение животных за пределы естественного ареала.

Экологическая опасность также создается в связи с развитием биотехнологии и генной инженерии. При несоблюдении санитарных норм возможно попадание из лаборатории или завода в окружающую природную среду микроорганизмов и биологических веществ, оказывающих весьма вредное воздействие на биотические сообщества, здоровье человека и его генофонд.

Помимо генно-инженерных аспектов, среди актуальных вопросов биобезопаности, имеющих важное значение для сохранения биоразнообразия, выделяют также:
\begin{itemize}
    \item перенос генетической информации от домашних форм к диким видам;
    \item генетический обмен между дикими видами и подвидами, в том числе риск генетического загрязнения генофонда редких и исчезающих видов;
    \item генетические и экологические последствия преднамеренной и непреднамеренной интродукции животных и растений.
\end{itemize}

Особую опасность представляет биологическое загрязнение среды возбудителями инфекционных и паразитарных болезней. Значительные изменения окружающей среды в результате антропогенного воздействия приводят к непредсказуемым последствиям в поведении популяций возбудителей и переносчиков опасных для человека и животных болезней. Основные источники происхождения возбудителей болезней инфекционных болезней следующие:
\begin{itemize}
    \item Ряд возбудителей являются продуктом сопряженной эволюции паразитов, полученных человеком от своих обезьяноподобных предков. Пример подобной эволюции --- возбудитель малярии.
    \item Большая группа инфекционных болезней получена человеком от животных. В процессе развития земледелия, одомашнивания скота, возникновения и развития общества человек постоянно соприкасался со многими бактериями, вирусами и другими паразитическими организмами с начала дикими, а за тем и домашних животных, часть из которых адаптировалась к человеческим популяциям.
    \item Значительная группа болезней человека сложилась вследствие адаптации диких, свободноживущих форм к организму человека. Так, например, в Юго-Восточной Азии в древности создались своеобразные сочетания природных и социальных условий: жаркий климат, медленно текущие реки с множеством рукавов и затонов, высокая плотность населения, интенсивное фекальное загрязнение водоемов, тесный контакт населения с водой. Все это способствовало частым пассажам водных вибрионов через организм человека, а в результате естественного отбора возникла новая болезнь --- холера.
    \item Источником происхождения инфекций являются непатогенные паразиты, населяющие организм человека. К ним относятся возбудители дизентерии, амебиаза, эшерихиозов, возбудители кокковых инфекций.
\end{itemize}

Часто источником инфекции может является почва, в которой постоянно обитают возбудители столбняка, ботулизма, газовой гангрены, некоторых грибковых заболеваний. В организм человека они могут попасть при повреждении кожных покровов, с немытыми продуктами питания, при нарушении правил гигиены. Болезнетворные микроорганизмы могут проникнуть в грунтовые воды и стать причиной инфекционных болезней человека. Поэтому воду из артезианских скважин, колодцев, родников необходимо перед питьем кипятить. Также водоемы наиболее сильно подвергаются загрязнению, что способствует бурному развитию фитопланктона и вызывает цветение воды. Вода, подвергшаяся биологическому загрязнению микроорганизмами, становится непригодной для питья. Попадание в водоемы патогенных вирусов вызывает отравления и опасные болезни.

\section{Актуальность биологического загрязнения}

Актуальность биологического загрязнения связана с наличием целого ряда медицинских и экологических проблем, а именно:
\begin{itemize}
    \item несмотря на обширную информацию о механизме действия микроорганизмов, их воздействие на состояние здоровья человека все еще полностью не контролируется;
    \item загрязненность воздушной среды микроорганизмами и пылью органического происхождения на предприятиях биотехнологии, текстильной промышленности, на животноводческих и птицеводческих комплексах и ряде других производств является важнейшим фактором, оказывающим негативное воздействие на здоровье работающих;
    \item регистрируется неуклонный рост заболеваний, вызванных условно-патогенными возбудителями, представителями обычной микрофлоры человека;
    \item наблюдается возникновение многочисленных поствакцинальных осложнений в связи с повышенной сенсибилизацией организма человека.
    \item возникают трудности при лечении многих заболеваний вследствие широкого распространения в окружающей среде антибиотикоустойчивых микроорганизмов;
    \item все возрастающие темпы урбанизации создают опасность недостаточного обезвреживания огромного количества сточных вод;
    \item развитие биотехнологий, перевод животноводства и птицеводства на промышленную основу привело к увеличению масштабов микробного загрязнения как воздуха рабочей зоны, так и окружающей среды, а следовательно к ухудшению состояния здоровья людей, работающих на таких предприятиях и проживающих в районах расположения этих производств.
\end{itemize}

\section{Последствия биологического загрязнения}

Влияние биологического загрязнения на отдельные организмы и всю экосистему сложно оценить из-за воздействия дополнительных характеристик: климата, образа жизни, рациона питания. Примеры положительного влияния: формирование иммунитета к новым болезням, улучшение физических характеристик вида. Примеры негативного влияния: новые болезни, уничтожение отдельных видов животных и растений.

Всего выделяется 3 группы последствий биологического заражения:
\begin{itemize}
    \item эпидемии и пандемии;
    \item изменения экосистемы;
    \item воздействие на воду, почву и воздух.
\end{itemize}

Под влиянием биологического заражения происходит изменение и перераспределение структуры растительности и деятельности живых организмов. Это приводит к изменениям во всех элементах экосистемы: воде, почве, атмосфере. Появление новых микроорганизмов в водоемах приводит к их заболачиванию, появлению рясы и тины, что вынуждает обитателей покидать ареал или приводит к их гибели. Это приводит к нарушению цепочки питания и проникновению опасных элементов в организм живых существ. Биологическое загрязнение может также привести к снижению естественности природоохранных территорий, неблагоприятным экономическим последствиям и воздействию на здоровье человека. В ряде случаев случайно переселенные в новые экосистемы животные или растения могут приносить большой ущерб сельскому хозяйству. По такому же принципу влияет атмосферное загрязнение от промышленных предприятий и работающего транспорта. Еще одним примером могут выступать пестициды в сельском хозяйстве, которые через почву и грунтовые воды заражают места обитания животных, а также растения, которые к этим удобрениям не устойчивы.

Биологическое загрязнение может вызывать неблагоприятные последствия на нескольких уровнях биологической организации:
\begin{itemize}
    \item индивидуальный организм (внутреннее загрязнение паразитами или патогенами);
    \item популяция (путем генетического изменения, то есть гибридизации ИА с местным видом);
    \item сообщество или биоценоз (путем структурных сдвигов, т.е. доминирования ИА, замены или уничтожения местных видов);
    \item среда обитания (путем изменения физико-химических условий);
    \item экосистема (путем изменения потока энергии и органического материала).
\end{itemize}

Чрезвычайно опасными являются биологические загрязнения, которые вызываются патогенными микроорганизмами. Недостаточно очищенные и обезвреженные бытовые сточные воды содержат большой комплекс патогенных микроорганизмов, вызывающих кожные, кишечные, глистные заболевания. Однако бактерии содержатся повсеместно, и жизнь на земле была бы невозможна без этих одноклеточных микроорганизмов, помогающих разлагать и утилизировать органические остатки. Они могут находиться в атмосфере, воде, почве, в теле других живых организмов, в том числе и в самом человеке. Наиболее опасны возбудители инфекционных заболеваний. Они имеют различную устойчивость в окружающей среде. Одни способны жить вне организма человека всего несколько часов: находясь в воздухе, в воде, на разных предметах, они быстро погибают. Другие могут жить в окружающей среде от нескольких дней до нескольких лет. Для третьих окружающая среда является естественным местом обитания. Для четвертых — другие организмы, например дикие животные, являются местом сохранения и размножения.

Более того, существуют непатогенные или условно-патогенные микроорганизмы, которые в качестве места обитания избрали организм человека. Так, полости рта, носа, толстого кишечника, влагалища являются местом обитания многих микроорганизмов, которые не только не  вредят человеку, но и стимулируют его защитные силы, способствуют перевариванию остатков пищи, вырабатывают витамины. Однако превышение нормы количества этих микроорганизмов при неблагоприятных для  человека условиях (изменение температуры окружающей среды, снижение  иммунитета и пр.) изменяет соотношение микроорганизмов, приводя к дисбактериозу.

Примерами заболеваний, которые подобные загрязнители вызывают у человека, являются:
\begin{itemize}
    \item чума;
    \item холера;
    \item брюшной тиф;
    \item сибирская язва;
    \item дизентерия;
    \item дифтерия;
    \item корь;
    \item ОРВИ;
    \item птичий грипп;
    \item ВИЧ-инфекция;
    \item гепатит;
    \item туберкулез;
    \item чесотка.
\end{itemize}

Воздействие на среду обитания или на отдельные организмы также происходит через внедрение в них новых микроорганизмов. Внедрение осуществляется намеренно или самопроизвольно из-за побочных реакций других процессов. Посторонние элементы влияют на рацион питания, восприимчивость живых существ к внешней среде, на взаимодействие между видами. Основные виды загрязнителей:
\begin{itemize}
    \item инфекции;
    \item вирусы;
    \item мутация;
    \item смена естественного ареала.
\end{itemize}

Спрогнозировать последствия данного вида загрязнений для флоры и фауны, а также для экономики и экологической обстановки невозможно. Между тем значительный урон природным экосистемам, хозяйствам наносят:
\begin{itemize}
    \item заболевания людей, животных, растений из-за стремительного распространения вирусных инфекций;
    \item размножение вирусов, бактерий, появление новых штаммов болезнетворных микроорганизмов;
    \item риск генетического загрязнения генофонда региона;
    \item нарушение равновесия естественных экосистем, потеря способности биогеоценозов (природных сообществ) к гомеостазу — саморегуляции. Вследствие этого происходит появление в сообществе нетипичных для него живых организмов. Чаще всего уже живущие на данной территории организмы не способны конкурировать с видами-пришельцами, из-за чего погибают или мигрируют. Это нарушает установившиеся в экосистеме связи и пищевые цепи, вызывает их разрушение;
    \item деградация природных сообществ приводит к исчезновению уникальных животных, растений региона, ведь именно эндемические виды особенно остро реагируют на изменение окружающей среды;
    \item красные приливы, которые связаны со стремительным размножением одноклеточных водорослей. Заражение морей и океанов токсичными организмами вызывает отравления людей, питающихся рыбой из пораженных водоемов. Красные приливы происходят все чаще, ведь они спровоцированы загрязнением водной среды.
\end{itemize}

\section{Методы борьбы с биологическим загрязнением}

По степени загрязнения выделяют 5 уровней биогенов – от 0 до 4. При этом учитывают:
\begin{itemize}
    \item количество нехарактерных видов;
    \item размер площади распространения;
    \item вред на экологию в загрязненном регионе;
    \item глобальное влияние биогенов.
\end{itemize}

Чтобы предупредить риск развития биологического заражения, нужно:
\begin{itemize}
    \item усилить контроль над безопасностью лабораторных работ;
    \item повсеместно соблюдать санитарно-гигиенические нормы;
    \item уделить внимание очищению сточных вод;
    \item контролировать состояние сооружений для очистки;
    \item проверять питьевую воду на пригодность;
    \item проводить вакцинацию среди населения;
    \item вводить при необходимости карантинные меры;
    \item контролировать вирусные очаги.
\end{itemize}

Перечисленные способы не принесут должного результата, если не устранить причину. Необходимо разрабатывать программу по предотвращению биологического загрязнения биосферы. Например, использование экологически чистых энергетических источников, применение штрафов к предприятиям, выбрасывающим отходы в окружающую среду. Бережное отношение к природе должно стать выгодным для всех жителей планеты.

Когда происходит биологическое загрязнение окружающей среды часто наблюдается усиленный рост вредных для организма человека микроорганизмов. Такой всплеск из активности зачастую приводит к различного рода заболеваниям и даже эпидемиям, не свойственным ранее для данной местности. И именно по этой причине сразу после биологического загрязнения необходимо провести целый спектр работ по его ликвидации. Разнообразие видов работ, связанных с опасностью вредного действия биологических факторов на организм людей, также очень велико. Их можно разделить на следующие группы:
\begin{enumerate}
    \item Связанные с вредным воздействием микроорганизмов и продуктов их жизнедеятельности, например:
          \begin{itemize}
              \item работы по производству и контролю биологических препаратов, основой или продуцентами которых являются микроорганизмы, биологические жидкости, ткани и органы, а также культуры клеток и тканей;
              \item работы по использованию биологических препаратов для профилактики, лечения, диагностики и других целей в медицине, ветеринарии и сельском хозяйстве;
              \item работы по локализации и ликвидации очагов инфекционных болезней;
              \item работы по использованию культур микроорганизмов в научно-исследовательских, учебных и практических учреждениях;
              \item работы, требующие соприкосновения с почвой и водой (местами возможного обитания микроорганизмов);
              \item работы по лечению и уходу за людьми и животными (больными и носителями);
              \item работы по исследованию материалов от людей и животных, а также трупного материала в диагностических и научно-исследовательских целях.
          \end{itemize}
    \item Связанные с опасностью вредного воздействия животных (домашних, диких и лабораторных) и продуктов их жизнедеятельности, например:
          \begin{itemize}
              \item работы по обслуживанию животных в сельском хозяйстве и при производстве биологических препаратов, продуцентами которых они служат;
              \item работы по обслуживанию животных в вивариях научно-исследовательских и практических учреждений;
              \item охотничьи и рыболовные промыслы;
              \item убой животных;
              \item переработка сырья животного происхождения;
              \item обслуживание и дрессировка животных в зоологических садах и цирках.
          \end{itemize}
    \item Связанные с опасностью вредного воздействия растений (культурных и дикорастущих), например:
          \begin{itemize}
              \item работы по выращиванию растений в сельском, лесном и городском хозяйствах;
              \item работы по сбору и переработке растительного сырья;
              \item лесохозяйственные работы и работы по заготовке леса;
              \item работы по производству лекарственных препаратов и аллергенов из растений;
              \item работы по производству кормов.
          \end{itemize}
\end{enumerate}

Все компоненты, входящие в структуру биологического фактора, разделяют на две основные группы:
\begin{enumerate}
    \item \textbf{Естественно-природные} --- это элементы естественных природных процессов, способные при определенных условиях играть роль биологического фактора, например:
          \begin{itemize}
              \item возбудители, переносчики и носители инфекционных заболеваний людей;
              \item возбудители инфекционных заболеваний животных;
              \item возбудители инфекционных заболеваний птиц;
              \item естественные отходы животного мира;
              \item пыльца при цветении растений;
              \item сине-зеленые водоросли;
              \item заплесневелые предметы.
          \end{itemize}
    \item \textbf{Техногенные} --- гигиенически значимые факторы (микроорганизмы, готовые продукты, пыль растительного происхождения и т. д.), присущие
          \begin{itemize}
              \item промышленно-животноводческим комплексам;
              \item производству и использованию микробиологических средств защиты растений;
              \item производству и использованию антибиотических средств;
              \item производству и использованию белково-витаминных концентратов;
              \item производствам по улучшению и использованию стимуляторов роста;
              \item сооружениям по очистке сточных вод;
              \item производствам сена, льна, хлопка, зерна;
              \item производствам вакцин и сывороток;
              \item биологическое загрязнение производствам физиологически активных препаратов.
          \end{itemize}
\end{enumerate}

Предотвращение и исправление последствий биозагрязнения достигается применением индивидуальных и общих мер. К индивидуальным относятся мероприятия, способствующие формированию устойчивого иммунитета к воздействию вредных микроорганизмов, соблюдение правил гигиены. К общим мерам относятся действия по государственному регулированию применения опасных веществ, проведение профилактических мероприятий, искусственное изменение экосистемы (например, формирование новых водоемов).

Среди мероприятий по борьбе с загрязнением выделяются 3 группы самых распространенных:
\begin{itemize}
    \item соблюдение правил гигиены;
    \item карантин;
    \item очистка вод.
\end{itemize}

Выделяют следующие способы борьбы с разрушениями:
\begin{itemize}
    \item регулирование численности населения,
    \item введение карантина (если возникает такая необходимость),
    \item регулярные наблюдения эколого-эпидемиологического характера,
    \item снижение очагов опасных вирусных заболеваний и контроль за ними,
    \item уменьшение возможного генетического загрязнения редких видов, занесенных в красную книгу,
    \item постоянный надзор эпидемиологов за развитием вирусов,
    \item санитарная охрана местности.
\end{itemize}

\section{Пример биологического загрязнения}

Одним из самых известных примеров биологического загрязнения является <<Долина смерти>> при основании вулкана Кихпиныч в восточной части полуострова Камчатка. Там в 1975 году зоолог В. Каляев и вулканолог В. Леонов впервые обнаружили случаи массовой гибели птиц и зверей. В ручье с мрачным названием Гибельный они заметили необычные крупные валуны серо-желтого цвета, похожие на спящих медведей. Подойдя поближе, они поняли, что не ошиблись, это действительно были медведи, только не спящие, а мёртвые. Их шерсть была покрыта небольшим налётом серы.

Вначале ученые посчитали, что гибель животных вызывает слишком высокое содержание сероводорода и углекислого газа в гейзерах, но все было не так просто: оба эти соединения не вызывают мгновенной смерти, а положение тел животных и наличие свободного выхода из долины показывали на то, что их гибель была делом нескольких секунд. При чем большинство трупов было сосредоточено в одной впадине 100 м в длину и около 50 м в ширину. После обнаружения <<Долины Смерти>> в ней были развернуты широкие исследовательские работы, которые велись на протяжении 8 лет — до 1983 года.

В ходе исследований удалось выяснить, что первыми жертвами смертельно опасного места ранней весной становились мелкие грызуны и птицы семейства воробьиных, потом в долине появлялись лисы, почуяв легкую добычу, и так-же погибали. Вслед за ними приходили росомахи, тоже в надежде поживиться павшими животными, однако долина становилась смертельной и для них. К разлагающимся трупам спускались вороны и беркуты, для которых этот роскошный обед тоже был последним.

Планомерное исследование принесло свои плоды: учеными было установлено, что гибель животных происходила как правило в весенне-летний период, после таяния снегов, при этом погибали в основном плотоядные животные, так как из-за отсутствия растительности травоядные практически не приходили в это смертельно опасное для жизни место. Наибольшая опасность поджидала и животных, и людей в пониженных местах этой долины, там, где концентрация смертельного газа была наибольшей. Как только исследователи спустились вниз, ко впадине с максимальным количеством погибших животных, как у них сразу же начинала болеть и словно гореть голова, у них начиналось сильное головокружение и слабость. Но стоило им подняться по склонам вверх, все болезненные симптомы весьма быстро проходили, потому именно серу и ее соединения на протяжении долгого времени продолжали считать основной причиной гибели всего живого в этой долине.

И лишь в 1982 году был выявлен смертоносный элемент, убивавший животных моментально, как было в случае с охотниками. Среди газовых выделений <<Долины Смерти>> на Камчатке удалось выделить цианистый водород и хлористый циан, которые вызывают мгновенный паралич дыхания и остановку сердца. Именно от них и погибали медведи, росомахи, рыси и сотни других животных, которые угодили в эту смертельную ловушку. Инстинкт самосохранения, который столь сильно развит у животных, в этом случае почему-то не срабатывал.

Долина Смерти на Камчатке, хоть и редкое, но не является единственным смертельно опасным природным явлением, связанным с вулканической деятельностью планеты. Подобные долины существуют в США, в Индонезии и в Италии. Причиной гибели в такого рода местах чаще становится смертельная концентрация углекислого газа, который вырывается на поверхность земли, а вот соединения циана присутствуют лишь в гейзерах Камчатки.

\section{Заключение}

Биологическое загрязнение --- это весьма значимая проблема, которая затрагивает множество аспектов жизни не только человека, но и живых существ в принципе. Чтобы решить проблему биологического загрязнения человечество должно более ответственно относиться к этой проблеме и делать все возможное, чтобы предотвратить ее дальнейшее развитие.

\section{Список использованных источников}

\begin{enumerate}
    \item Коробкин В.И., Передельский Л.В. Экология. - 12-e издание изд. - Ростов-на-Дону: Еникс, 2007. - 603 с.
    \item Хавкина Т. К. Антропогенное изменение окружающей среды и здоровья человека. - Саратов: Научная книга, 2009. - 442 с.
    \item Кленова И. А. Экология: учебное пособие. - Ростов-на-Дону: ФГБОУ ВО РГУПС, 2017. - 259 с.
    \item Долина Смерти. Причина гибели всего живого // Livejournal URL: \url{https://terrao.livejournal.com/8629638.html} (дата обращения: 19.05.2023).
\end{enumerate}

\end{document}
